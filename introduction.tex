% Литературный обзор с целью описания контекста интеграции моей статьи в общий объем научных работ

% Необходимо ответить на вопросы:
% 1. Что уже известно в данной области исследования
% 2. Какие аспекты остатся недостаточно изученными
% 3. Какой вклад предполагается внести в научное сообщество

% 1000 слов

% 3 ссылки на литературу

% Факт с проблематикой
% Подтверждение актуальности
% Анализ того, как сейчас решается проблема
% Определение места того, в чем существующие решения слабы, где у них пробелы и белые места, в чем их надо доработать



% Статья: 1
% Графовая модель применяется для: семантического анализа программ
% Целью приведенного в статье алгоритма является: автоматизация тестирования
% В алгоритме для описания работы с графом используютя структуры данных: деревья

% Статья 2
% Графовая модель применяется для: выявления применимости механизмов многопоточных вычислений в программе
% Целью приведенного в статье алгоритма является: выявление циклов
% В алгоритме для описания работы с графом используютя структуры данных: многомерные массивы

% Статья 3
% Графовая модель применяется для: оптимизации кода
% Целью приведенного в статье алгоритма является: оптимизация паралельного кода
% В алгоритме для описания работы с графом используютя структуры данных: матрицы
% Примечание: Описанные модели оень похожи на те, что у меня. При написании стаьи вернуться и посмотреть, что можно написать таакого, чтобы их применить для моей задачи

% Статья 4
% Графовая модель применяется для: описания работы статического анализатора программного кода
% Целью приведенного в статье алгоритма является: построение графа
% В алгоритме для описания работы с графом используютя структуры данных: структуры

% Статья 5
% Графовая модель применяется для: маршрутизации потоков в сети
% Целью приведенного в статье алгоритма является: поиск кратчайших путей
% В алгоритме для описания работы с графом используютя структуры данных: List<Структур>

% Статья 6
% Графовая модель применяется для: маршрутизации потоков в сети
% Целью приведенного в статье алгоритма является: поиск кратчайших путей
% В алгоритме для описания работы с графом используютя структуры данных: структуры

% Статья 7
% Графовая модель применяется для: анализа используемых дв вычислительной системе ресурсов
% Целью приведенного в статье алгоритма является: выявление простаиваемых системных ресурсов
% В алгоритме для описания работы с графом используютя структуры данных: структуры

% Статья 8
% Графовая модель применяется для: генерации состояний программного компонента
% Целью приведенного в статье алгоритма является: 
% В алгоритме для описания работы с графом используютя структуры данных: 

% Статья 9
% Графовая модель применяется для: коррекции значений счетчиков
% Целью приведенного в статье алгоритма является: 
% В алгоритме для описания работы с графом используютя структуры данных:

% Статья 10
% Графовая модель применяется для: описания связей распределенной системы
% Целью приведенного в статье алгоритма является: разрезание циклов
% В алгоритме для описания работы с графом используютя структуры данных: массивы

% Статья 11
% Графовая модель применяется для: описание работы многопоточной работы
% Целью приведенного в статье алгоритма является: 
% В алгоритме для описания работы с графом используютя структуры данных: 

% Статья 12
% Графовая модель применяется для: демонстрации работы исследуемого алгоритма
% Целью приведенного в статье алгоритма является: 
% В алгоритме для описания работы с графом используютя структуры данных:

% Статья 13
% Графовая модель применяется для: описания алгоритма обхода графа в грубину
% Целью приведенного в статье алгоритма является: 
% В алгоритме для описания работы с графом используютя структуры данных:

% Статья 14
% Графовая модель применяется для: 
% Целью приведенного в статье алгоритма является: описание графовых математических моделей
% В алгоритме для описания работы с графом используютя структуры данных: списки

% Статья 15
% Графовая модель применяется для: 
% Целью приведенного в статье алгоритма является: 
% В алгоритме для описания работы с графом используютя структуры данных: используется язык запросов

% Статья 16
% Графовая модель применяется для: множество решаемых задач
% Целью приведенного в статье алгоритма является: 
% В алгоритме для описания работы с графом используютя структуры данных: язык запросов
% ! очень похоже на то, что делаю я. ссылаться на эту работу при аргументации своихх мыслей

% Статья 17
% Графовая модель применяется для: поиска оптимальной конфигурации для работы вычислительной системы
% Целью приведенного в статье алгоритма является: 
% В алгоритме для описания работы с графом используютя структуры данных: массивы

% Статья 18
% Графовая модель применяется для: поиска оптимальной конфигурации для работы вычислительной системы
% Целью приведенного в статье алгоритма является: 
% В алгоритме для описания работы с графом используютя структуры данных: массивы

% Статья 19
% Графовая модель применяется для: решение задачи декомпозиции
% Целью приведенного в статье алгоритма является: поиск кратчайших путей в графе
% В алгоритме для описания работы с графом используютя структуры данных: массивы

% Статья 20
% Графовая модель применяется для: проектирование интерфейсов классов
% Целью приведенного в статье алгоритма является: 
% В алгоритме для описания работы с графом используютя структуры данных: vector, set


Графовые математические модели нашли широкое применение при решение прикладных задач программирования в самом широком спектре. В настоящее время графовые модели используются для исследования структур компьютерных программ в самых различных аспектах: семантический анализ, иерархическое описание, граф вызова процедур и т.д. Это обусловлено простотой описания актуальных для программирования проблем в терминах графов, а значит они могут быть решены при помощи идентичных хорошо изученных и математически обоснованных механизмов из теории графов.

При описании работы потоков управления, потоков данных, связи составных частей распределенной системы очень важно указывать направление потока или связи. В этих вопросах всегда есть источник, а есть потребитель. Иногда один компонент системы, обозначаемый на графове узлом, является одновременно источником и потребителем. Кроме того, он может быть источником для нескольких потребителей и потребителем от нескольких источников. В ряде задач один узел может быть источником и потребителем для самого себя, например в случае наличия обратных связей в описываемой моделе.


