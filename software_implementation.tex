При достаточно большом количестве элементов изначального графа трассируемости требований на файлы исходного кода не представляется возможным его преобразование и последующая оценка эффективности без применения инструментов предоставляемых вычислительной техникой и возможностей языков программирования. Не существует универсального способа описания графа на языке программирования с целью применения полученной реализации для решения произвольной задачи. В каждом отдельном случае, для каждой отдельно взятой задачи существует необходимость описания графовой модели и алгоритмов действий над ней. Однако существуют распространенные практики, позволяющие эффективно описывать графовые математические модели на тех или иных языках программирования.

% Теперь про практики

Так, в работах ... для описания состава узлов и связей между ними применяются массивы целочисленных данных. В работах ... продемонстрировано использование динамических списков. Широкое применение для решения такого класса задач нашли структуры, их возможности описаны в ... .