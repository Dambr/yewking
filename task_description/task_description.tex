\documentclass{article}
\usepackage{tempora}
\usepackage{indentfirst}
\usepackage{tabularx}
\usepackage{caption}
\usepackage{graphicx}
\usepackage{longtable}
\usepackage{tabularx}
\usepackage{amsmath}
\usepackage{amsfonts}
\usepackage{floatrow}
\floatsetup[table]{capposition=top}
\makeatletter
\graphicspath{ {./images/} }
\renewcommand*\l@section{\@dottedtocline{1}{1.5em}{2.3em}}
\makeatother
\usepackage{float}
\usepackage[english, russian]{babel}
\begin{document}
  \textbf{Постановка задачи}.

  Пусть имеется $n$ требований к ПО, которые трассируются на $m$ файлов исходного кода. Файлы исходного кода распределны по $k$ плагинам и имеют $l$ типов связностей друг с другом.

  Программное решение поставляется в виде набора плагинов. Каждый из плагинов включает в поставку файлы исходного кода, которые в среде выполнения реализуют некоторый функционал. Поставленный функционал формируется исходя из потребностей заказчика. Он определяет минимум функционала, который должен быть реализован в поставке - вектор полезных для заказчика требований ($R^{u}$). Каждое из возможных требований проидентифицировано натуральным числом $i = 1, 2, \cdots, n$. $R^{u}_{i}$ принимает значение $0$ если $i$-е требование в рамках поставки для заказчика бесполезно. $1$ - если требование полезно в рамках поставки ($0 \le r^{u} \le 1, r^{u} \in \mathbb{Z}$).

  Трассируемость требования на файлы исходного кода предполагает, что каждое требование реализовано по меньшей мере в одном файле, образовывая связь \textit{<<требование - файл>>} (матрица $E^{rf}$). Значение элементов $e^{rf}$ соответствует условной доле участия файла в реализации требования ($0 \le e^{rf} \le 1, e^{rf} \in \mathbb{R}$). Каждое требование реализовано полностью, поэтому выполняются условия:
  \begin{center}
    $\displaystyle\sum^{m}_{i = 1} e^{rf}_{1, i} = 1, \displaystyle\sum^{m}_{i = 1} e^{rf}_{2, i} = 1, \cdots, \displaystyle\sum^{m}_{i = 1} e^{rf}_{n, i} = 1$
  \end{center}

  Распределение файлов по плагинам предполагает, что каждый файл расположен в одном из плагинов, образовывая связь \textit{<<файл - плагин>>} (матрица $E^{fp}$). Значение элемента $e^{fp}$ может быть равным $0$, если файл не расположен в плагине, или $1$, если он расположен ($0 \le e^{fp} \le 1, e^{fp} \in \mathbb{Z}$). Все файлы распределены по плагинам, при этом один и тот же файл не может быть расположен одновременно в двух разных плагинах, поэтому выполняются условия:
  \begin{center}
    $\displaystyle\sum^{k}_{i = 1} e^{fp}_{1, i} = 1, \displaystyle\sum^{k}_{i = 1} e^{fp}_{2, i} = 1, \cdots, \displaystyle\sum^{k}_{i = 1} e^{fp}_{m, i} = 1$
  \end{center}

  Связность файлов друг с другом предполагает, что каждый файл может быть зависим от других файлов $l$ различными способами. В каждом из способов образуется связь \textit{<<файл - файл>>} (матрицы $E^{ff_{1}}, E^{ff_{2}}, \cdots, E^{ff_{l}}$). Значение элемента $e^{ff}$ может быть равным $0$, если связь отсутствует, или $1$, если связь присутствует ($0 \le e^{ff} \le 1, e^{ff} \in \mathbb{Z}$). Связь между файлами считается разрешенной в рамках одного типа связности, если разрешена связь между файлами на всей глубине цепочки зависимостей по этой связности.

  Требуется вычислить такие значения элементов $e^{fp}$, чтобы при заданных $n$, $m$, $k$, $l$, $E^{rf}$, $E^{ff_{1}}$, $E^{ff_{2}}$, $\cdots$, $E^{ff_{l}}$ минимизировать значение коэффициента бесполезности. Коэффициент бесполезности прямо пропорционален разности объемов поставляемых требований и полезных, и обратно пропорционален объему поставляемых требований.

  Для описания целевой функции рассматривается следующий алгоритм решения задачи:
  \begin{enumerate}
    \item Считать полезные требования
    \item Вычислить полезные файлы исходного кода
    \item Разрешить зависимости между файлами исходного кода и сформировать объем файлов, который должен быть поставлен
    \item Определить перечень плагинов, в которые включены файлы
    \item Сформировать перечень файлов, которые входят в состав плагинов
    \item Сформировать перечень требований, которые реализуются поставляемым объемом файлов исходного кода
    \item Вычислить значение коэффициента бесполезности в данной поставке
  \end{enumerate}

  Тогда целевую функцию можно описать так:

  \begin{center}
    $\frac{\displaystyle\sum A_{all}\Bigg(E^{rf} \cdot A_{some}\bigg(E^{fp} \cdot A_{some}\Big(\big(\sum^{l}_{i}\sum^{m}_{j}D^{i}(j)\big) \cdot E^{fp}\Big)\bigg)\Bigg) - \sum R^{u}}{\displaystyle\sum A_{all}\Bigg(E^{rf} \cdot A_{some}\bigg(E^{fp} \cdot A_{some}\Big(\big(\sum^{l}_{i}\sum^{m}_{j}D^{i}(j)\big) \cdot E^{fp}\Big)\bigg)\Bigg)}$
    % $R^{d} = activate_r(R^{td})$
    % $R^{td} = E^{rf} \cdot F^{d}$
    % $F^{d} = activate_f(F^{td})$
    % $F^{td} = E^{fp} \cdot P^{d}$
    % $P^{d} = activate_p(P^{td})$
    % $P^{td} = F^{dep} \cdot E^{fp}$
    % $F^{dep} = \sum^{l}_{i}\sum^{m}_{j} D^{i}(j)$
    % $D^{i}(j) = 
    %   \begin{cases}
    %     E^{ff_{i}} \cdot F^{u} & \quad \text{если } j = 1
    %     E^{ff_{i}} \cdot D^{i}(j - 1) & \quad \text{если } j > 1
    %   \end{cases}
    % $
    % $F^{u} = R^{u} \cdot E^{rf}$
  \end{center}

  где:
  \begin{itemize}
    \item[] $A_{all}(x) = 
              \begin{cases}
                1 & \quad \text{если } x = 1 \\
                0 & \quad \text{если } x < 1
              \end{cases}
            $
    \item[] $A_{some}(x) = 
              \begin{cases}
                1 & \quad \text{если } x > 0 \\
                0 & \quad \text{если } x = 0
              \end{cases}
            $
    \item[] $D^{i}(x) = 
              \begin{cases}
                E^{ff_{i}} \cdot (R^{u} \cdot E^{rf}) & \quad \text{если } x = 1 \\
                E^{ff_{i}} \cdot D^{i}(x - 1) & \quad \text{если } x > 1
              \end{cases}
            $
  \end{itemize}
\end{document}
