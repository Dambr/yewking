\documentclass{article}
\usepackage{tempora}
\usepackage{indentfirst}
\usepackage{tabularx}
\usepackage{caption}
\usepackage{graphicx}
\usepackage{longtable}
\usepackage{tabularx}
\usepackage{amsmath}
\usepackage{amsfonts}
\usepackage{floatrow}
\floatsetup[table]{capposition=top}
\makeatletter
\graphicspath{ {./images/} }
\renewcommand*\l@section{\@dottedtocline{1}{1.5em}{2.3em}}
\makeatother
\usepackage{float}
\usepackage[english, russian]{babel}
\begin{document}
  Нагрузка по матрицам. Необходимо сделать так, чтобы потом решен можно было расширить до генерации не одной матрицы, а плодить их в больших количествах.

  Можно вообще в файл их писать (json) например.

  Что я подаю на вход?
  Набор чисел:
  \begin{itemize}
    \item число комплектаций
    \item минимальное число полезных требований в комплектации
    \item максимальное число полезных требований в комплектации
    \item число требований
    \item число число файлов исходного кода
    \item число плагинов
    \item минимальное число зависимостей у файла
    \item максимальное число зависимостей у файла
    \item минимальная цена вхождения требования в поставку
    \item максимальная цена вхождения требования в поставку
  \end{itemize}

  Что я ожидаю получить на выходе?
  Набор матриц:
  \begin{itemize}
    \item список комплектаций
    \item матрица стоимостей
    \item матрица трассируемости требований на файлы
    \item матрица зависимостей между файлами
  \end{itemize}

  На что направлен функционал этой генерации?
  Варианты:
  \begin{itemize}
    \item поиск такого объема параметров в моделе, на котором glpk начнет ломаться
    \item сбор информации о различном потребном для работы времени glpk на различных объемах параметров в модели
  \end{itemize}

  
  
\end{document}
