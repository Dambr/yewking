\documentclass{article}
\usepackage{tempora}
\usepackage{indentfirst}
\usepackage{tabularx}
\usepackage{caption}
\usepackage{graphicx}
\usepackage{longtable}
\usepackage{tabularx}
\usepackage{amsmath}
\usepackage{amsfonts}
\usepackage{floatrow}
\floatsetup[table]{capposition=top}
\makeatletter
\graphicspath{ {./images/} }
\renewcommand*\l@section{\@dottedtocline{1}{1.5em}{2.3em}}
\makeatother
\usepackage{float}
\usepackage[english, russian]{babel}
\begin{document}
  \thispagestyle{empty}
  
  Целевая функция:
  \begin{itemize}
    \item[] $\displaystyle min \sum (E^{rr} \cdot R^{d}_{t}) \quad (dim = n \times 1)$
    \item[] $R^{d}_{t} = Cim(E^{rf} \cdot F^{d}_{t}) \quad (dim = n \times 1)$
    \item[] $F^{d}_{t} = Cin(E^{fp} \cdot (P^{d})^{T}) \quad (dim = m \times 1)$
    \item[] $P^{d} = F^{dep} \cdot E^{fp} \quad (dim = 1 \times k)$
    \item[] $\displaystyle F^{dep} = F^{u} + \sum^{m - 1}_{i = 0}D(i) \quad (dim = 1 \times m)$
    \item[] $F^{u} = R^{u} \cdot E^{rf} \quad (dim = 1 \times m)$
  \end{itemize}

  Вспомогательные функции:
  \begin{itemize}
    \item[] $
    Cim(x) = 
      \begin{cases}
        0 & \quad \text{если } x < 1 \\
        1 & \quad \text{если } x = 1
      \end{cases}
    $
    \item[] $
      Cin(x) =
      \begin{cases}
        0 & \quad \text{если } x = 0 \\
        1 & \quad \text{если } x > 0
      \end{cases}
    $
    \item[] $
      D(x) =
      \begin{cases}
        F^{u} \cdot E^{ff} & \quad \text{если } x = 0 \quad (dim = 1 \times m)\\
        D(x - 1) \cdot E^{ff} & \quad \text{если } x > 0 \quad (dim = 1 \times m)
      \end{cases}
    $
  \end{itemize}

  Используемые обозначения:
  \begin{itemize}
    \item[] $n$ - кол-во требований
    \item[] $m$ - кол-во файлов
    \item[] $k$ - кол-во плагинов
    \item[] $R^{u} = 1 \times n$ - вектор маркерных значений о полезности требований
    \item[] $E^{rr} = n \times n$ - матрица стоимостей реализованных требований в поставке
    \item[] $E^{rf} = n \times m$ - матрица трассируемости требований на файлы исходного кода
    \item[] $E^{ff} = m \times m$ - матрица зависимостей файлов исходного кода друг относительно друга
    \item[] $E^{fp} = m \times k$ - матрица распределения файлов исходного кода по плагинам. Рассчет ее элементов необходимо произвести
  \end{itemize}
  
  На элементы матрицы $E^{fp}$ действуют следующие ограничения:
  \begin{center}
    $\displaystyle \sum^{k}_{i}e^{fp}_{1, i} = 1, \sum^{k}_{i}e^{fp}_{2, i} = 1, \cdots, \sum^{k}_{i}e^{fp}_{m, i} = 1$
  \end{center}

\end{document}
