% Литературный обзор с целью описания контекста интеграции моей статьи в общий объем научных работ

% Необходимо ответить на вопросы:
% 1. Что уже известно в данной области исследования
% 2. Какие аспекты остатся недостаточно изученными
% 3. Какой вклад предполагается внести в научное сообщество

% 1000 слов

% 3 ссылки на литературу

% Факт с проблематикой
% Подтверждение актуальности
% Анализ того, как сейчас решается проблема
% Определение места того, в чем существующие решения слабы, где у них пробелы и белые места, в чем их надо доработать



% Статья: 1
% Графовая модель применяется для: семантического анализа программ
% Целью приведенного в статье алгоритма является: автоматизация тестирования
% В алгоритме для описания работы с графом используютя структуры данных: деревья

% Статья 2
% Графовая модель применяется для: выявления применимости механизмов многопоточных вычислений в программе
% Целью приведенного в статье алгоритма является: выявление циклов
% В алгоритме для описания работы с графом используютя структуры данных: многомерные массивы

% Статья 3
% Графовая модель применяется для: оптимизации кода
% Целью приведенного в статье алгоритма является: оптимизация паралельного кода
% В алгоритме для описания работы с графом используютя структуры данных: матрицы
% ! Описанные модели очень похожи на те, что у меня. При написании стаьи вернуться и посмотреть, что можно написать таакого, чтобы их применить для моей задачи

% Статья 4
% Графовая модель применяется для: описания работы статического анализатора программного кода
% Целью приведенного в статье алгоритма является: построение графа
% В алгоритме для описания работы с графом используютя структуры данных: структуры

% Статья 5
% Графовая модель применяется для: маршрутизации потоков в сети
% Целью приведенного в статье алгоритма является: поиск кратчайших путей
% В алгоритме для описания работы с графом используютя структуры данных: List<Структур>

% Статья 6
% Графовая модель применяется для: маршрутизации потоков в сети
% Целью приведенного в статье алгоритма является: поиск кратчайших путей
% В алгоритме для описания работы с графом используютя структуры данных: структуры

% Статья 7
% Графовая модель применяется для: анализа используемых дв вычислительной системе ресурсов
% Целью приведенного в статье алгоритма является: выявление простаиваемых системных ресурсов
% В алгоритме для описания работы с графом используютя структуры данных: структуры

% Статья 8
% Графовая модель применяется для: генерации состояний программного компонента
% Целью приведенного в статье алгоритма является: 
% В алгоритме для описания работы с графом используютя структуры данных: 

% Статья 9
% Графовая модель применяется для: коррекции значений счетчиков
% Целью приведенного в статье алгоритма является: 
% В алгоритме для описания работы с графом используютя структуры данных:

% Статья 10
% Графовая модель применяется для: описания связей распределенной системы
% Целью приведенного в статье алгоритма является: разрезание циклов
% В алгоритме для описания работы с графом используютя структуры данных: массивы

% Статья 11
% Графовая модель применяется для: описание работы многопоточной работы
% Целью приведенного в статье алгоритма является: 
% В алгоритме для описания работы с графом используютя структуры данных: 

% Статья 12
% Графовая модель применяется для: демонстрации работы исследуемого алгоритма
% Целью приведенного в статье алгоритма является: 
% В алгоритме для описания работы с графом используютя структуры данных:

% Статья 13
% Графовая модель применяется для: описания алгоритма обхода графа в грубину
% Целью приведенного в статье алгоритма является: 
% В алгоритме для описания работы с графом используютя структуры данных:

% Статья 14
% Графовая модель применяется для: 
% Целью приведенного в статье алгоритма является: описание графовых математических моделей
% В алгоритме для описания работы с графом используютя структуры данных: списки

% Статья 15
% Графовая модель применяется для: 
% Целью приведенного в статье алгоритма является: 
% В алгоритме для описания работы с графом используютя структуры данных: используется язык запросов

% Статья 16
% Графовая модель применяется для: множество решаемых задач
% Целью приведенного в статье алгоритма является: 
% В алгоритме для описания работы с графом используютя структуры данных: язык запросов
% ! очень похоже на то, что делаю я. ссылаться на эту работу при аргументации своихх мыслей

% Статья 17
% Графовая модель применяется для: поиска оптимальной конфигурации для работы вычислительной системы
% Целью приведенного в статье алгоритма является: 
% В алгоритме для описания работы с графом используютя структуры данных: массивы

% Статья 18
% Графовая модель применяется для: поиска оптимальной конфигурации для работы вычислительной системы
% Целью приведенного в статье алгоритма является: 
% В алгоритме для описания работы с графом используютя структуры данных: массивы

% Статья 19
% Графовая модель применяется для: решение задачи декомпозиции
% Целью приведенного в статье алгоритма является: поиск кратчайших путей в графе
% В алгоритме для описания работы с графом используютя структуры данных: массивы

% Статья 20
% Графовая модель применяется для: проектирование интерфейсов классов
% Целью приведенного в статье алгоритма является: 
% В алгоритме для описания работы с графом используютя структуры данных: vector, set

% Какую задачу я решаю
Получение сведений о трассируемости требований к программному обеспечению (ПО) на файлы исходного кода приложения является сложной прикладной проблемой. Зачастую такие сведения необходимы для анализа эффективности проведенных работ по проектированию приложения, оценке стоимости его тестирования~[1], а так же управления его конфигурацией. В данной работе предложен способ пострения схемы, описывающей в каких файлах исходного кода реализуется заданный объем требований к ПО и адаптированной для получения сведений о зависимостях файлов исходного кода между собой.

% Для решения задач описания ПО используются графовые математические модели
Для решения задач, связанных с оптимизацией ПО~[17], получением данных о загруженности ресурсов вычислительной системы~[7], иерархическим описанием зависимостей компонентов~[10] широкое применение нашли графовые математические модели. Это обусловлено простотой описания актуальных для программирования проблем с использованием графовых математических моделей~[12].

% Описание работы потоков
При описании работы потоков управления, потоков данных, связи составных частей распределенной системы очень важно указывать направление потока или связи~[5],~[6]. При организации связей всегда есть источник и есть потребитель. Иногда один компонент системы, обозначаемый на графе вершиной, является одновременно источником и потребителем. Кроме того, он может быть источником для нескольких потребителей и потребителем от нескольких источников. В ряде задач одна вершина может быть источником и потребителем для самого себя, например в случае наличия обратных связей в описываемой модели. Для описания таких связей используются ориентированные графы, в которых по направлению дуг можно судить о принадлежности вершины к числу источников или приемников.

% Задачи оптимизации ПО
Для решения задач, связанных с оптимизацией многопоточного ПО~[3], необходим механизм преобразования графа из исходного вида к целевому. Преобразование происходит по заранее сформулированным правилам и может осуществляться за несколько итераций. Условие окончания проведения итераций преобразования так же определено заранее. Для решения задачи построения графа трассируемости требований к ПО на файлы исходного необходимо, чтобы вне зависимости от очередности вершин графа, к которым применяются действия по преобразованию, результирующий граф всегда формировался бы одинаково. Описанный в статье~[3] способ не гарантирует этого. 

Изложенный в работе~[19] подход к формированию графа с применением алгоритмов нейронных сетей~[11] не учитывает ограничение, что вершины результирующего графа должны быть двух категорий. В то же время приведенный способ формирования результирующего графа может быть доработан для решения задачи построения оптимальной декомпозиции компонентов приложения с целью максимизации вариантов комплектаций его поставки при заданном трассировании требований к ПО на файлы исходного кода с учетом разрешенных циклических зависимостей и выполняя свою работу на уже предварительно преобразованном графе.

Описанные в~[3] и~[15] приемы преобразования ориентированного графа и видоизменения его в результате итерационно выполняемых действий нацелены на построение такого графа, который бы упрощал поиск цепочки задействованных в одном сценарии работы ПО вершин графа. Такие приемы требуют доработки и адаптации для решения задачи построения графа трассируемости требований к ПО на файлы исходного кода с учетом возможности наличия циклических зависимостей у файлов исходного кода между собой.

Кроме того, при описании способов преобразования исходного графа необходимо учитывать временные издержки, которые появляются в ходе выполнения операций над исходным графом. Объем издержек возрастает, если в исходном графе присутствует значительное число требований к ПО и файлов исходного кода.
