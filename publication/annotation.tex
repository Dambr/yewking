% Цели и задачи
% Краткое описание:
% 1. Методологии исследования
% 2. Результатов
% 3. Актуальность и значимость исследования

% 150-200 слов

Получение сведений о трассируемости требований к программному обеспечению (ПО) на файлы исходного кода приложения является сложной прикладной проблемой. Особенно при установлении необходимого объема верификационных процедур, выполнение которых необходимо для подтерждения выполнимости требований к ПО при изменении одного или нескольких файлов исходного кода. \textbf{Цель:} разработка модели, которая бы обеспечивала пользователя информацией о связности файлов исходного кода между собой и требованиями к ПО. \textbf{Результаты:} модель разработана. Цель достигается за счет разрешения циклических зависимостей и формирования графа трассируемости требований к ПО на общность файлов исходного кода, неразрывно связанных друг с другом. Предложена программная реализация, в состав которой включены опциональные модули для исследования ее оптимальной конфигурации. Исследована зависимость времени работы от реализации способа хранения данных в разработанном программном решении. \textbf{Практическая значимость:} разработанная модель позволяет получать полный перечень требований к ПО, корректность выполнения которых не может быть гарантирована после внесения изменений в файлы исходного кода разрабатываемого приложения. \textbf{Обсуждение:} в дальнейшем планируется использовать разработанную модель и ее программную реализацию для решения задачи оптимальной декомпозиции файлов исходного кода по плагинам инструментального средства конфигурирования, выполненного как часть плагинной системы с целью исследования и разработки принципов построения таких программных решений.

\textbf{Ключевые слова} - требования к ПО, файлы исходного кода, графовые математические модели, ориентированный граф, циклы.