% Формулируются выводы сделанные на основе проведенного исследования
% Удалось ли реализовать поставленные задачи и достигнуть целей
% Подчеркнуть вклад исследования в развитие научной области, какие перспективы открывают полученные данные для научного сообщества
% Кто и каким образом сможет применить новые знания на практике в будущем
% Какие моменты в исследовании нуждаются в дополнительной проработке или углабленно анализе

% 2 ссылки на литературу

В статье предложена модель, отображающая трассируемость требований к ПО на файлы исходного кода и позволяющая разрешать циклические зависимости. Используя ее можно получать информацию о степени связности файлов исходного кода друг с другом, на какую долю функционала оказывается влияние при внесении изменений в файлы исходного кода. Данные сведения могут быть полезны, например, при оценке необходимого объема выполнения верификационных процедур после внесения изменений в один или несколько файлов исходного кода.

В дальнейших исследованиях предполагается применение результатов работы сформированной модели с целью решения задачи оптимальной декомпозиции программного решения инструментального средства конфигурирования выполненного как модуль плагинной системы с целью распределения файлов исходного кода по плагинам для максимизации числа комплектаций поставки разработанного программного решения.