% Факт и проблема
На сегодняшний день в области информационных технологий актуален вопрос поиска оптимального соотношения цена/полезность. При этом востребовано как уменьшение цены при сохранении полезности, так и увеличение полезности при сохранении цены. Здесь под полезностью понимается способность информационного продукта удовлетворять потребности заказчика. Зачастую информационный продукт поставляется как есть и приобретается или предоставляется в пользование заказчику включая как полезные, так и бесполезные для заказчика функциональные возможности. Объем и характер функционала влияют на стоимость продукта. Управление включаемым функционалом и возможность исключения из продукта бесполезного функционала может понизить стоимость без потери полезности продукта. Это повышает конкурентоспособность продукта на рынке информационных технологий по сравнению с существующими конкурентными продуктами.

% Существует решение
Для обеспечения управления конфигурацией и настройки конечного объема доступного для заказчика функционала можно использовать технологию плагинных систем. В плагинной системе на результирующий объем функционала влияют установленные в такой системе расширения - плагины. Применение плагинов для реализации механизма исключения бесполезного функционала из решения не лишено недостатка. В этой части основным недостатком выделяется необходимость поставки бесполезного функционала в силу существования функциональных зависимостей.

% Гипотеза, как применять
Была выдивинута гипотеза, что поиск оптимальной декомпозиции файлов исходного кода, реализующих функционал программного решения, по плагинам, находящимся под конфигурационным контролем, может быть осуществлен при помощи технологии искусственного интеллекта, а именно с применением генетического алгоритма.

% Что нужно знать, чтобы ее реализовать
Для построения математической модели необходимо учесть следующую информацию о моделируемой системе:
\begin{enumerate}
    \item Какой функционал предполагается поставлять в рамках каждой из заявленных комплектаций.
    \item Общий объем функциональных требований.
    \item Объем кодовой базы.
    \item Предполагаемое допустимое системой управления конфигурацией число плагинов.
    \item Трассируемость требований к ПО на файлы исходного кода.
    \item Функциональные зависимости между файлами исходного кода.
    \item Правило ценообразования поставки в зависимости от реализованных в ней функциональных требований.
\end{enumerate}

% Про математическую модель (что нового и в чем достоинство)
Поиск оптимальной декомпозиции может быть осуществлен с применением прогргаммных решателей если сформулировать целевую функцию и ограничения на значения параметров построенной математической модели. Однако если сформулированная целевая функция и ограничения к модели являются нелинейными выражениями, то число решателей, способных обработать модель значительно сокращается. Кроме того, многие решатели являются проприетарными и применимы по открытой лицензии только при ограниченном числе параметров и ограничений в модели. Описанное в настоящей работе решение лишено данного недостатка, т.к. использует технологии с открытым исходным кодом, распространяемые по открытой лицензии, а так же и не имеет явных ограничений на размер обрабатываемой математической модели.

