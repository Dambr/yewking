% Для наваливания математики нужно в формулах расписать кроссенговер, выбор родителя, сохранение и несохранение предыдущих поколений

% Расписать конкретные настройки генетического алгоритма
% Как минимум привести ссылки на различные алгоритмы и обосновать применение того или другого используемого алгоритма
% 

% Построенная модель - сослаться на написанную статью
% Формирование стоимости - сослаться на готовую статью
% Fitness функция - описать
% Про представление матрицы бинарных отношений в виде вектора
% Выявление недостатка и описание перехода к дискретным величинам

\subsection*{Используемая модель}
Работа генетического алгоритма производится на моделе, описанной в [1]:
\begin{itemize}
  \item используются те же входные данные;
  \item стоимость комплектаций вычисляется по той же формуле без приведения к линейному виду вследствие отсутствия необходимости работы с линейными выражениями;
\end{itemize}

С целью сохранения способности модели отвечать условию, что один файл не может одновременно находиться в разных плагинах, в работе [1] существуют ограничения на искомые значения матрицы. Работа генетического алгоритма не предполагает наличие таких ограничений, поэтому для стимуляции эволюции решений в сторону соблюдения данного условия применяется проверка значений в каждом элементе полученной матрицы.

\subsection*{}

\subsection*{Работа с бинарными данными}

\subsection*{Валидация решения}

\subsection*{Работа с дискретными данными}

\subsection*{Оценка эффективности}
Оценка эффективности полученного в процессе работы генетического алгоритма решения производится по сследующему алгоритму:
\begin{enumerate}
  \item образование вектора значений;
  \item конвертация вектора значений в матрицу бинарных отношений;
  \item валидация матрицы;
  \item если матрица валидна, то:
  \begin{itemize}
    \item вычисление суммарной стоимости комплектаций от применения решения [1];
    \item результирующее значение fitness функции равно обратному значению вычисленной суммарной стоимости.
  \end{itemize}
  \item если матрица невалидна, то результирующее значение fitness функции равно $0$.
\end{enumerate}
