% Опишите, как ваши значения элементов матрицы будут кодироваться (например, бинарные строки или действительные числа).
\subsection*{Представление решения}
В рамках исследования пригодности использования генетического алгоритма для решения задачи [1] реализовано два решения:
\begin{enumerate}
  \item обработка бинарных значений;
  \item обработка дискретных значений.
\end{enumerate}

При описании обоих решений используются следующие величины из [1]:
\begin{enumerate}
  \item $m$ - число файлов исходного кода в версии ПО;
  \item $k$ - число плагинов, распределение файлов по которым необходимо осуществить.
  \item $X_{m \times k}$ - в настоящей работе так обозначена матрица распределения файлов по плагинам $A_{m \times k}$.
\end{enumerate}

Обработка бинарных или дискретных значений предусматривает подбор значений соответствующих векторов:
\begin{enumerate}
  \item $\dot{X}$ - вектор бинарных значений, имеет длину $m \cdot k$;
  \item $\hat{X}$ - вектор целочисленных значений в диапазоне $[0; k]$, имеет длину $m$.
\end{enumerate}

В процессе работы необходимо трансформировать вектора в матрицу $X$:
\begin{enumerate}
  \item при обработке $\dot{X}$:
  \begin{center}
    $X_{i, j} = \dot{X}_{i \cdot k + j - k} \quad i=\overline{1, m} \quad j=\overline{1, k}$
  \end{center}
  \item при обработке $\hat{X}$:
  \begin{center}
    $X = Z$ \\
    $j = \hat{X}_{i} \quad X_{i, j} = 1 \quad i=\overline{1, m}$
  \end{center}
\end{enumerate}

% Укажите размер популяции и как инициируются начальные решения (рандомно или с использованием эвристик).
\subsection*{Начальная популяция}
При формировании начальной популяции не используются эвристики, она инициализируется случайным образом используя следующие входные данные:
\begin{enumerate}
  \item количество решений в популяции равно $4$;
  \item количество генов в каждом решении соответствует длине вектора, обработка которого осуществляется алгоритмом.
\end{enumerate}

% Опишите вашу формулу, объясните, как она будет использоваться для оценки качества решений (функция приспособленности)
\subsection*{Функция оценки}
Оценка качества решения может быть осуществлена только для валидного решения. Валидным называется решения, для которого выполняется условие:
\begin{center}
  $\displaystyle \sum X_{i} = 1 \quad i=\overline{1, m}$
\end{center}

Верно, что при обработке $\dot{X}$ невалидное решение может быть получено.

Доказательство: используемая формула не гарантирует наличие хотябы одного элемента, равного $1$ в результирующем решении, равно как не исключает появления нескольких элементов, равных $1$ при сколь угодно великих размерах результирующей матрицы $X$.


Верно, что при обработке векторов $\dot{X}$ и $\hat{X}$ невалидное решение может быть получено для $\dot{X}$ и не может быть получено для $\dot{X}$. Доказательство:

наличие одного и только одного элемента равного $1$, а всех остальных $0$ в каждой строке матрицы $X$.

Оценка качества решения - величина, обратно пропорциональная результирующей стоимости, полученной в результате применения решения, сформированного в процессе работы алгоритма:
\begin{center}
  $\displaystyle 1 / \sum f_{c}$
\end{center}


\begin{center}
  $
  \displaystyle fitness =
  \begin{cases}
    1 / \sum f_{c} & \quad \text{если решение валидно} \\
    0 & \quad \text{если решение невалидно}
  \end{cases}
  $
\end{center}

% Подробно опишите, как будут применяться селекция, скрещивание и мутация.
% Укажите параметры, такие как вероятность мутации и способ скрещивания.
\subsection*{Операторы ГА}

% Уточните, какие условия будут использованы для остановки алгоритма (например, максимальное количество поколений или достижение заданного уровня приспособленности).
\subsection*{Условия завершения}

% Упомяните о возможных модификациях ГА, которые могут улучшить результаты
\subsection*{Потенциальные улучшения}


