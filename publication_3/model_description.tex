% Опишите, как ваши значения элементов матрицы будут кодироваться (например, бинарные строки или действительные числа).
\subsection*{Представление решения}
В рамках исследования пригодности использования генетического алгоритма для решения задачи [1] реализовано два решения:
\begin{enumerate}
  \item по обработке бинарными величинами;
  \item по обработке дискретными величинами.
\end{enumerate}

При описании обоих решений используются следующие величины из [1]:
\begin{enumerate}
  \item $m$ - число файлов исходного кода в версии ПО;
  \item $k$ - число плагинов, распределение файлов по которым необходимо осуществить.
  \item $X$ - в настоящей работе так обозначена матрица распределения файлов по плагинам $A$.
\end{enumerate}

Обработка величин предусматривает подбор значений соответствующих векторов:
\begin{enumerate}
  \item $\dot{X}$ - вектор бинарных величин, имеет длину $m \cdot k$;
  \item $\hat{X}$ - вектор целочисленных значений в диапазоне $[0; k]$, имеет длину $m$.
\end{enumerate}

В процессе работы необходимо трансформировать вектора в матрицу $X$:
\begin{enumerate}
  \item при обработке $\dot{X}$:
  \begin{center}
    $X_{i, j} = \dot{X}_{i \cdot k + j - k} \quad i=\overline{1, m} \quad j=\overline{1, k}$
  \end{center}
  \item при обработке $\hat{X}$:
  \begin{center}
    $X = Z$ \\
    $j = \hat{X}_{i} \quad X_{i, j} = 1 \quad i=\overline{1, m}$
  \end{center}
\end{enumerate}

% Укажите размер популяции и как инициируются начальные решения (рандомно или с использованием эвристик).
\subsection*{Популяция}
Размер 

% Опишите вашу формулу, объясните, как она будет использоваться для оценки качества решений (функция приспособленности)
\subsection*{Функция оценки}

% Подробно опишите, как будут применяться селекция, скрещивание и мутация.
% Укажите параметры, такие как вероятность мутации и способ скрещивания.
\subsection*{Операторы ГА}

% Уточните, какие условия будут использованы для остановки алгоритма (например, максимальное количество поколений или достижение заданного уровня приспособленности).
\subsection*{Условия завершения}

% Упомяните о возможных модификациях ГА, которые могут улучшить результаты
\subsection*{Потенциальные улучшения}


