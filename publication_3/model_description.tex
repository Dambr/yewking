% Нужно в общих чертах изложить, что лежит в основе реализации генетического алгоритма, а так же что такой взгляд позволяет решить задачу двумя различными способами

% Вопрос: нужно ли заново расписывать, как считается стоимость? Думаю нет. Нужно ссылаться на работу опубликованную.

% Что нужно расписать:
% Фитнес функцию
% Бинарный конфиг
% Дискретный конфиг

% Не нужно писать про граф. Нужно раскрыть, как происходит конвертация.

\subsection*{Формирование модели}
Учитывая информацию о моделируемой системе формируется образ будущей модели: модель описывает элементы и связи между ними. Так образуется граф.

Вершинами графа являются:
\begin{enumerate}
    \item функциональные требования;
    \item файлы исходного кода;
    \item плагины.
\end{enumerate}

Вершины в графе располагаются слоями. Связи присутствуют между слоями и между элементами одного слоя, но не могут пересекать какой-либо слой. Так, в графе связи между вершинами образуются от:
\begin{enumerate}
    \item трассирования требований на файлы исходного кода;
    \item зависимостей файлов исходного кода между друг другом;
    \item распределения файлов исходного кода по плагинам;
    \item влияния наличия реализованного в поставке требования на ее стоимость.
\end{enumerate}

На рисунке приведен пример графа.

\subsection*{Входные и выходные данные}
Входными данными для построения математической модели являются:
\begin{enumerate}
    \item $k$ - число плагинов
    \item $l$ - число комплектаций
    \item $m$ - число файлов исходного кода
    \item $n$ - число функционалных требований
    \item $T_{n \times m}$ - матрица трассируемости требований на файлы исходного кода
    \item $D_{m \times m}$ - матрица зависимости файлов исходного кода между друг другом
    \item $C_{n \times n}$ - матрица стоимостей включения требований в комплектацию
    \item $E_{l \times n}$ - матрица полезных требований в заявленных комплектациях
\end{enumerate}

% Я сейчас сижу и переписываю то, что и так в препринте описал
% Варианта 2:
% 1. Немного другими словами то же самое описать
% 2. Дождаться публикации в препринтах и сослаться на ту работу

Значения матрицы $T_{n \times m}$ описывают степень вовлеченности $i$-го файла в реализацию $j$-го требования. Если файл не вовлечен в реализацию требования, то $t_{i, j} = 0$. В противном случае $0 < t_{i, j} \le 1$. При этом:
\begin{center}
  $\displaystyle \sum^{m}_{j}t_{i, j} = 1, \quad i = \overline{1, n}$
\end{center}

В матрице бинарных отношений $D_{m \times m}$ описаны зависимости между файлами исходного кода. Если у $i$-го файла присутствует зависимость на $j$-й, то $d_{i, j} = 1$, иначе $d_{i, j} = 0$. В рамках оптимизационной задачи полагается, что файл не имеет зависимости самого на себя. В силу этого значения элементов на главной диагонале матрицы равны $0$.

Значения матрицы $C_{n \times n}$ указывают, на сколько изменится цена комплектации, если при реализованном $i$-м требовании так же будет реализовано и $j$-е.

Выходными данными модели являются значения матрицы бинарных отношений $A_{m \times k}$. В ней описывается распределение файлов по плагинам. Если $j$-й плагин включает $i$-й файл $a_{i, j} = 1$, иначе $a_{i, j} = 0$. При этом:
\begin{center}
    $\displaystyle \sum^{k}_{j} a_{i, j} = 1 \quad i = \overline{1, n}$
\end{center}

\subsection*{Оценка оптимальности}
Оптимальность сгенерированного алгоритмом решения обратно пропорционально его стоимости.

\subsection*{Стратегия работы алгоритма с бинарными данными}

\subsection*{Стратегия работы алгоритма с дискретными данными}

\subsection*{Функция пригодности}
Работа генетического алгоритма при решении оптимизационной задачи может быть оценена благодаря использованию функции пригодности. Функция пригодности призвана в количественном отношении оценить степень оптимальности предложенного алгоритмом решения. В своей работе генетический алгоритм 

