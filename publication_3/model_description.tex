% Опишите, как ваши значения элементов матрицы будут кодироваться (например, бинарные строки или действительные числа).
\subsection*{Представление решения}
В рамках исследования пригодности использования генетического алгоритма для решения задачи [1] реализовано два решения:
\begin{enumerate}
  \item обработка бинарных значений;
  \item обработка дискретных значений.
\end{enumerate}

При описании обоих решений используются следующие величины из [1]:
\begin{enumerate}
  \item $m$ - число файлов исходного кода в версии ПО;
  \item $k$ - число плагинов, распределение файлов по которым необходимо осуществить.
  \item $X_{m \times k} = ||x_{i, j}||$ - в настоящей работе так обозначена матрица распределения файлов по плагинам.
\end{enumerate}

Обработка бинарных или дискретных значений предусматривает подбор значений соответствующих векторов:
\begin{enumerate}
  \item $\dot{X}_{1 \times m \cdot k} = ||\dot{x}_{i}||$ - вектор бинарных значений;
  \item $\hat{X}_{1 \times m} = ||\hat{x}_{i}||$ - вектор целочисленных значений в диапазоне $[0; k]$.
\end{enumerate}

В процессе работы необходимо трансформировать вектора в матрицу $X$:
\begin{enumerate}
  \item при обработке $\dot{X}$:
  \begin{center}
    $x_{i, j} = \dot{x}_{i \cdot (k - 1) + j} \quad i=\overline{1, m} \quad j=\overline{1, k}$
  \end{center}
  \item при обработке $\hat{X}$:
  \begin{center}
    $X = Z_{m \times k}$ \\
    $j = \hat{x}_{i} \quad x_{i, j} = 1 \quad i=\overline{1, m}$
  \end{center}
\end{enumerate}

% Опишите вашу формулу, объясните, как она будет использоваться для оценки качества решений (функция приспособленности)
\subsection*{Функция оценки}
Оценка качества решения может быть осуществлена только для валидного решения. Валидным называется решения, для которого выполняется условие:
\begin{center}
  $\displaystyle \sum X_{i} = 1 \quad i=\overline{1, m}$
\end{center}

Заметим, что метод обработки $\hat{X}$ гарантирует получение валидной матрицы $X$. В то время как метод обработки $\dot{X}$ этого не гарантирует.

Если $X$ валидна, то может быть вычислено значение результирующей стоимости $cost$ с использованием формулы, приведенной в работе [1].

Тогда значение пригодности решения $fitness$:
\begin{center}
  $
  \displaystyle fitness =
  \begin{cases}
    1 / cost & \quad \text{если решение валидно} \\
    0 & \quad \text{если решение невалидно}
  \end{cases}
  $
\end{center}

\subsection*{Стратегия эволюции}
% Укажите размер популяции и как инициируются начальные решения (рандомно или с использованием эвристик).
При формировании начальной популяции не используются эвристики, она инициализируется случайным образом используя следующие входные данные:
\begin{enumerate}
  \item в каждой популяции при работе генетического алгоритма задействовано $4$ решения;
  \item количество генов в каждом решении соответствует длине вектора, обработка которого осуществляется алгоритмом.
\end{enumerate}

% Подробно опишите, как будут применяться селекция, скрещивание и мутация.
% Укажите параметры, такие как вероятность мутации и способ скрещивания.
При реализации генетического алгоритма использованы следующие типы операций:
\begin{enumerate}
  \item селекции - steady state selection;
  \item кроссовера - scattered;
  \item мутации - scramble;
\end{enumerate}

На основе $2$-х лучших родительских решений формируются $2$ дочерних решения. Лучшие родительские решения так же используются на следующей итерации работы генетического алгоритма.

% Уточните, какие условия будут использованы для остановки алгоритма (например, максимальное количество поколений или достижение заданного уровня приспособленности).
Работа генетического алгоритма завершается после смены $100$ поколений.
