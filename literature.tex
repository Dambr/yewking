\begin{enumerate}
    \item А.Н. Вигура, анализ и тестирование программ на основе алгебраической модели, Информационные технологии, Вестник Нижегородского университета им. Н.И. Лобачевского, 2011, No 5 (1), с. 185–190;
    \item А.М. Шульженко, автоматическое определение циклов ParDo в программе, Естественные науки, известия ВУЗов. северо-кавказский регион, ISSN 0321-3005, с. 77-87;
    \item B. П. Корячко, д-р техн. наук проф., C. В. Скворцов, канд. техн. наук доц., Иерархическая модель глобальной оптимизации у параллельных объектных программ, электронный журнал "Инженерное образование", 2006; 
    \item Кошелев В.К., Игнатьев В.Н., Борзилов А.И. Инфраструктура статического анализа программ на языке C\#. Труды ИСП РАН, том 28, вып. 1, 2016 г., с. 21-40;
    \item А. А. Чертков, Я. Н. Каск, Л. Б. Очина, Маршрутизация потоковой сети на основе модификации алгоритма Беллмана - Форда, ФГБОУ ВО «ГУМРФ имени адмирала С. О. Макарова», Санкт-Петербург, Российская Федерация, 2022г, топ 14 № 4, с 615-627;
    \item В. В. Сахаров, А. А. Чертков, Л. Б. Очина, Маршрутизация сетей с отрицательными весами звеньев в пакете оптимизации MATLAB ФГБОУ ВО «ГУМРФ имени адмирала С. О. Макарова», Санкт-Петербург, Российская Федерация, 2019г, том 11 № 2, с 230-242;
    \item К.В. Недоводеев, Метод генерации графов потоков данных, используемых при автоматическом синтезе параллельных программ для неоднородных многоядерных процессов, Научно-технические ведомости СПбГПУ 3' 20122 Информатика. Телекоммуникации. Управление, с 47-52;
    \item Ю. И. Евсеева, А. C. Бождай, Метод структурно-параметрического синтеза адаптивных программных компонентов виртуальной образовательной среды, Известия высших учебных заведений. Поволжский регион, DOI 10.21685/2072-3059-2016-3-8, с 84-92;
    \item О.А. Четверина, Методы коррекции профильной информации в процессе компиляции, Труды ИСП РАН, том 27, вып. 6, 2015 г. с 49-65;
    \item О.Б. Штейнберг, Минимизация количества временных массивов в задаче разбиения циклов, ISSN 0321-3005 известия ВУЗов, Северо-Кавказский регион, естественные науки, 2011. № 5, с 31-35;
    % \item Н. В. Заборовский, А. Г. Тормасов, докт. физ.‑мат. наук, Моделирование многопоточного исполнения программы и метод статического анализа кода на предмет состояний гонки, 
    \item Тарков М. С., Об эффективности построения гамильтоновых циклов в графах распределенных вычислительных систем рекуррентными нейронными сетями, Информационные технологии в управлении, Институт физики полупроводников им. А.В. Ржанова СО РАН, Новосибирск, Управление большими системами. Выпуск 43, 2013, с 157-171;
    \item С.В. Огородов, Обоснование линейноупорядоченного представления графовых моделей программ, Институт «Кибернетический центр» ТПУ, Известия Томского политехнического университета. 2008. Т. 312. № 5, с 85-89;
    \item Фролов А. С., канд. техн. наук Семенов А. С, Обзор проблемно-ориентированных языков программирования для параллельного анализа статических графов, Computational nanotechnology 1-2017, ISSN 2313-223X, 27-32;
    \item Е. П. Емельченков, В. И. Мунерман, Д. В. Мунерман, Т. А. Самойлова, Один метод построения циклов в графе, Современные информационные технологии и ИТ-образование. 2021. Т. 17, № 4. С. 814-823 
    \item А.А. Каленкова, Оптимизация потоков работ по времени выполнения, основанная на удалении избыточных потоков управления, ТРУДЫ МФТИ. — 2009. — Том 1, № 2, 160-174;
    \item А.П. Баглий, Н.М. Кривошеев, Б.Я. Штейнберг, О.Б. Штейнберг, Преобразования программ в Оптимизирующей распараллеливающей системе для распараллеливания на распределенную память, Инженерный вестник Дона, №12 (2022), ivdon.ru/ru/magazine/archive/n12y20225/8089;
    \item О.Б. Штейнберг, И.А. Ивлев, Применение преобразования циклов "Retiming" с целью уменьшения количества используемых регистров, Южный федеральный университет, г. Ростов-на-Дону, Россия, ISSN 0321-2653 известия ВУЗов. Северо-Кавказский регион. Технические науки. 2017. № 3, с 76-80;
    \item А. Ю. Попов, Применение вычислительных систем смногими потоками команд и одним потоком данных для решения задач оптимизации, ISSN 0236-3933. Вестник МГТУ им. Н.Э. Баумана. Сер. "Приборостроение", 2012, с 146-154; 
    \item Карпов Ю.Л., Волкова И.А., Вылиток А.А., Карпов Л.Е., Сметанин Ю.Г. Проектирование интерфейсов классов графовой модели нейронной сети. Труды ИСП РАН, том 31, вып. 4, 2019 г., стр. 97-112.
\end{enumerate}