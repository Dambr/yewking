\begin{enumerate}
    \item А.Н. Вигура, анализ и тестирование программ на основе алгебраической модели, Информационные технологии, Вестник Нижегородского университета им. Н.И. Лобачевского, 2011, No 5 (1), с. 185–190;
    \item А.М. Шульженко, автоматическое определение циклов ParDo в программе, Естественные науки, известия ВУЗов. северо-кавказский регион, ISSN 0321-3005, с. 77-87;
    \item B. П. Корячко, д-р техн. наук проф., C. В. Скворцов, канд. техн. наук доц., Иерархическая модель глобальной оптимизации у параллельных объектных программ, электронный журнал "Инженерное образование", 2006; 
    \item Кошелев В.К., Игнатьев В.Н., Борзилов А.И. Инфраструктура статического анализа программ на языке C\#. Труды ИСП РАН, том 28, вып. 1, 2016 г., с. 21-40;
    \item А. А. Чертков, Я. Н. Каск, Л. Б. Очина, Маршрутизация потоковой сети на основе модификации алгоритма Беллмана - Форда, ФГБОУ ВО «ГУМРФ имени адмирала С. О. Макарова», Санкт-Петербург, Российская Федерация, 2022г, топ 14 № 4, с 615-627;
    \item В. В. Сахаров, А. А. Чертков, Л. Б. Очина, Маршрутизация сетей с отрицательными весами звеньев в пакете оптимизации MATLAB ФГБОУ ВО «ГУМРФ имени адмирала С. О. Макарова», Санкт-Петербург, Российская Федерация, 2019г, том 11 № 2, с 230-242;
    \item К.В. Недоводеев, Метод генерации графов потоков данных, используемых при автоматическом синтезе параллельных программ для неоднородных многоядерных процессов, Научно-технические ведомости СПбГПУ 3' 20122 Информатика. Телекоммуникации. Управление, с 47-52;
    \item 
    \item
    \item
    \item
    \item
    \item
    \item
    \item
    \item
    \item
    \item
    \item
    \item
\end{enumerate}