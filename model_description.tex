% Требование на модель
Это достигается благодаря графу трассируемости требований на исходный код с количественной оценкой циклически зависимых друг на друга файлов исходного кода.

В случае описания зависимостей по направлению звеньев из узла и в узел можно судить о том, на какие какие файлы исхожного кода существует зависимость у текущего, а так же для каких файлов он является зависимостью.

В случае описания трассируемости начальными вершинами будут являться файлы исходного кода, а конечными вершинами - требования, реализация которых описана в соответствующих файлах исходного кода.

(рисунок)

Построенный граф отображает, как реализуются требования в файлах исходного кода программы. Заметим, что в звенья данного графа удобно разделить на две группы: группа требований и группа файлов исходного кода. Звенья графа из группы требований не имеют звеньев между друг другом. Кроме того, одно требование может быть реализовано в нескольких файлах исходного кода. Звенья графа из группы файлов исходного кода имеют звенья между друг другом, так же связи могут образовывать цикл. Однако нет необходимости указывать связь файла на самого себя, так как для цели отображения файлов исходного кода, которые должны войти в сборку такая связь не имеет смысл.

Объединение звеньев, имеющих друг на друга циклические зависимости, следует производить итерационно до тех пор пока в результирующем графе не будут отсутствовать циклы. После объединения узлов в одну группу граф видоизменяется. При этом для звеньев, которые объединяются выполняется:
\begin{itemize}
	\item звенья между друг другом удаляются
	\item узлы объединяются в группу
	\item образованная группа является новым звеном графа, при этом:
	\begin{itemize}
		\item образованная группа является начальным узлом для звеньев между конечными узлами для звеньев, связывающих объединенные узлы
		\item образованная группа является конечным узлом для звеньев между начальными узлами для звеньев, связывающих объединенные узлы
	\end{itemize}
\end{itemize}

При итерационном поиске циклов в графе вышеобозначенные группы учитывются как обыычные узлы и таким образом могут учавствовать в образовании новых групп по вышеуказанным правилам.