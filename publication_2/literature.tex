% Какая литература нужна:
% 1. О функциональных зависимостях
% 2. О разделяемых ресурсах
% 3. Об иерархии классов
% 4. х3 О ценообразовании и важности учета стоимости программного продукта
% 5. х3 О факторах, влияющих на стоимость ППО
% 6. х2 О плагинных системах
% 7. х3 О плагинах
% 8. х3 О составных частях плагинной системы


% 1 - Про требования и их трассировку. Отсутствуют выходные данные публикации
% 2 - Про оптимизацию вычислений на процессоре Intel
% 3 - Про решению прикладной задачи при помощи графов
% 4 - Про решение прикладной задачи при помощи графов
% 5 - Не моя тематика
% 6 - Не моя тематика
% 7 - Описание задачи в виде целевой функции, ограничений и метод big M
% 8 - Актуальность работы с постпродажным обслуживанием
% 9 - Не моя тематика
% 10 - Было в 5
% 11 - Было в 6
% 12 - Было в 7
% 13 - Поиск клонов в программном коде
% 14 - Описание компилятора Kotlin

% Вывод: Можно использовать 2 работы: 1 и 7. 7 - под вопросом

% 15 - плагины без зависимостей
% 16 - разные авспекты программного обеспечения (надежность, скорость и т.д.)
% 17 - про организацию вычислительных ресурсов в системе (микросервисы)
% 18 - про новый способ (подход) к программированию
% 19 - про понижение стоимости в информационной безопасности
% 20 - про стоимость информационных систем
% 21 - про стоимость информационных систем
% 22 - про решатели и pyomo
% 23 - про решатели

\begin{enumerate}
  
  \item \label{lit:7} И.П. Богданов, В.А. Нестеров, В.А. Судаков, К.И. Сыпало, Н. Б. Топоров, Расчет оптимальной загрузки воздушных транспортных средств с учетом приоритизации летательных аппаратов, известия ран. Теория и системы управления, 2021, № 3, с. 57–70
  \item \label{lit:15} Е.А. Тюменцев, Д.Ю. Згуровец, Алгоритм загрузки плагинов, не имеющих явных зависимостей между собой, Математические структуры и моделирование 2021. № 1(57). С. 101–107
  \item \label{lit:17} Р.О. Костромин, Сравнительный обзор средств управления конфигурациями ресурсов вычислительной среды функционирования цифровых двойников, «Information and mathematical technologies in science and management» 2021 № 1 (21)
  \item \label{lit:18} Д.С. Косарев, Д.Ю. Булычев, Обобщенное программирование с комбинаторами и объектами, Научно-технический вестник информационных технологий, механики и оптики, сентябрь–октябрь 2021, Том 21 № 5
  \item \label{lit:19} В.В. Василенко, С.В. Рыженко, Организация безопасного файлового обмена между корпоративной сетью и сетью общего пользования, Ежеквартальный рецензируемый, реферируемый научный журнал «Вестник АГУ». Вып. 2 (321) 2023
  \item \label{lit:20} А. Е. Архипов, С. В. Карпушкин, Структурно-параметрический синтез систем визуализации для тренажерных комплексов, Вестник Тамбовского государственного технического университета 2022. Том 28. № 3
  \item \label{lit:21} К.М. Демушкина, А.В. Кузьмин, Анализ возможностей инструментов реализации технологии process mining, Известия Самарского научного центра Российской академии наук, т. 25, № 4, 2023
  \item \label{lit:22} Сивакова Т.В., Судаков В.А., Шимко В.С. Исследование методов решения задач смешанного целочисленного линейного программирования // Препринты ИПМ им. М.В.Келдыша. 2024. № 24. 18 с.
  \item \label{lit:23} Белозеров И.А., Судаков В.А. Машинное обучение с подкреплением для решения задач математического программирования // Препринты ИПМ им. М.В.Келдыша. 2022. № 36. 14 с.
  \item \label{lit:24} Г.В. Добрянский, Н.С. Мельникова, В.Н. Мовила, Инструментальная программная платформа для разработки информационно-диагностических комплексов ИПП «Салют», Вестник УГАТУ, 2022, Т. 26, № 3 (97)
  \item \label{lit:25} Лаврищева Е.М., Зеленов С.В. Модельный подход к обеспечению безопасности и надежности Web-сервисов. Труды ИСП РАН, том 32, вып. 5, 2020 г., стр. 153-166.
\end{enumerate}