Оптимизация стоимостных издержек является актуальной проблемой в современных задачах разработки и сопровождения программного обеспечения. Данная проблема многогранна. В работе рассмотрена оптимизация стоимости постпродажного обслуживания. В качестве предметной области выбраны плагинные системы и решается задача оптимальной декомпозиции функционала при условии его поставки в разных комплектациях. Разработана математическая модель, которая обеспечивает оптимизацию распределения файлов исходного кода по плагинам по критерию минимальной стоимости сопровождения реализуемых ими требований. Предложена программная реализация модели на языке программирования Python с задействованием модуля Pyomo. Выполнена серия экспериментов по решению задачи оптимизации на одной и той же моделе различными решателями с целью выявления оптимальных значений искомых параметров и замера потребного времени на работу решателей.

\textbf{\textit{Ключевые слова}}: плагины, функционал, оптимизация, целевая функция, ограничения, решатели, Python, Pyomo.