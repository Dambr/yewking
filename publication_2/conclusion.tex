В статье предложена математическая модель для решения задачи оптимальной декомпозиции в предметной области построения программных решений для их выполнения в плагинной среде. Используя ее можно получать оптимальное распределение файлов исходного кода по плагинам с целью минимизации стоимости постпродажного обслуживания реализованного функционала в поставке заданной комплектации. Данные сведения могту быть полезны как при планировании архитектуры приложения, так и при формировании заявок комплектаций готовых версий программного обеспечения.

В дальнейших исследованиях предполагается изучение зависимости потребного времени работы решателей от объема модели: количества параметров и ограничений в ней. А так же изучение влияния на объем модели изменений в исходных данных задачи: изменение численных коэффициентов $l$, $n$, $m$ и $k$.
