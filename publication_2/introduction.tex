% Что актуально? О чем в дальнейшей работе? Чему она будет посвящена?
Одной из основных характеристик, оказывающих влияние на рентабильность программного обеспечения (ПО), вляется стоимость его постпродажного обслуживания (ППО). Формирование этого показателя может зависить от ряда факторов: начиная от особенностей самого ПО и заканчивая особенностями взаимодействия с его потенциальным заказчиком. В данной работе рассматривается случай, когда стоимость постпродажного обслуживания ПО зависит от числа реализованного в нем функционльных требований. При этом рассматривается ПО, которое может быть поставлено в разных комплектациях (отключаемое ПО [ссылка на ГОСТ 178]), а стоимость ППО каждого отдельного требования не постоянна, а зависит от состава поставки: наличие или отсутствие смежных требований может как увеличивать, так и уменьшать стоимость ППО.

% Какию математическую модель будем использовать?
Для решения задач, связанных с компонентами системы и анализом связей между ними, целесообразно применять графовые модели. 


% По каким маркерным признакам поняли, что именно эту модель целесообразно использовать?

% Что мы будем делать с моделью? Будет ли она изменяться? Что именно в ней будет изменяться? Где мы видели подобные механизмы?

% Какая схемы работы не подходит (с обоснованием почему)

% Какая еще схема работы не подходит (с обоснованием почему)

% Какой аспект нужно держать в голове при проектировании модели? Что ее усложняет? Какие параметры модели могут быть увеличены (численно)? Как это повлияет на модель?

