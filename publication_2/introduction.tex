На сегодняшний день актуален поиск оптимальных решений в сфере информационных технологий. Проанализировав работы [\ref{lit:19}], [\ref{lit:20}] и [\ref{lit:21}] можно сделать вывод, что оптимизировать необходимо не только техническую реализацию актуальных задач [\ref{lit:15}, \ref{lit:17}], но и стоимость решения как программного продукта. Снижение его стоимости может достигаться разными способами.
 % В настоящей работе рассматривается достижение стоимости за счет оптимизации расходов на его постпродажное обслуживание (ППО).

Анализ показал, что стоимость решения может зависеть от доступности и стоимости применяемых технологий [\ref{lit:21}], характера [\ref{lit:19}] реализации программного обеспечения (ПО), а так же от требований к квалификации оператора для взаимодействия с ПО [\ref{lit:20}]. Так, например, дополняющий друг друга функционал может снижать стоимость решения [\ref{lit:15}]. Повышать стоимость могут, например, конкурирующие за разделяемый ресурс единицы функционала [\ref{lit:17}].

% Про плагины
Была выдвинута гипотеза, что снижение стоимости может быть достигнуто за счет оптимизации расходов на постпродажное обслуживание (ППО). В рамках оптимизации из поставки решения исключается бесполезный для заказчика функционал. Такой способ был бы актуален для решений, которые динамически формируют конечный функционал программного комплекса, вносят в него новый и изменяют существующий. Примером таких решений являются плагинные системы [\ref{lit:24}, \ref{lit:25}]. В них интеграционными функциональными единицами являются плагины [\ref{lit:15}, \ref{lit:24}, \ref{lit:25}].

% На какие вопросы рассматриваемые ранее задачи ответили
После изучения работ [\ref{lit:24}] и [\ref{lit:25}], был сделан вывод, что в решаемой задаче необходимо учесть:
\begin{itemize}
  \item число требований - в них описываются функциональные возможности ПО
  \item число файлов исходного кода - в них на языке программирования реализованы требования
  \item число плагинов - они включают в себя файлы исходного кода
  \item трассируемость требований на файлы исходного кода
  \item зависимости между файлами исходного кода
  \item распределение файлов исходного кода по плагинам
\end{itemize}

% На какие вопросы рассматриваемые ранее задачи не ответили
Благодаря этим сведениям можно построить метаматическую модель, отвечающую на вопрос, какие требования будут реализованы в поставке. Однако, если имеющуюся информацию дополнить сведениями о стоимости сопровождения требований и механизмами формирования результирующей стоимости, тогда может быть решена задача оптимизации внутренней структуры программного решения по критерию минимальной стоимости ППО.
