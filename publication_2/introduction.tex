% Какие аналогичные задачи решаются по смежной тематике
Описание актуальных проблем в сфере информационных технологий в целом и области разработки ПО в частности на языке математики и составление математических моделей выполняется для широкого спектра проблем. В работе [] рассматривается решение задачи для организации зависимостей и решения проблемы dependency hell. Организация расписания использования разделяемых ресурсов описывается в работе []. Пример решения задачи об иерархии использования наследования классов приведен в работе []. 

Проанализировав работы [], [] и [] можно сделать вывод, что оптимизировать необходимо не только техническую реализацию актуальных задач, но и стоимость решения как программного продукта. Снижение его стоимости может достигаться разными способами. В настоящей работе рассматривается достижение стоимости за счет оптимизации расходов на его постпродажное обслуживание (ППО).

Анализ показал, что стоимость ППО зависит от объема [], характера [] и сложности организации комплексной работы функционала [], реализованного в ПО. Так, например, дополняющий друг друга функционал может снижать стоимость ППО []. Повышать стоимость ППО могут, например, конкурирующие за разделяемый ресурс единицы функционала [].

% Про плагины
Была выдвинута гипотеза, что снижения стоимости ППО может быть достигнуто путем исключения из решения незаявленного (бесполезного) для заказчика функционала. Такой способ был бы актуален для решений, которые динамически формируют конечный функционал программного комплекса, вносят в него новый функционал и изменяют существующий. Примером таких решений являются плагинные системы [], []. В них интеграционными функциональными единицами являются плагины [], [], [].

% На какие вопросы рассматриваемые ранее задачи ответили
После изучения работ [], [] и [], был сделан вывод, что в решаемой задаче необходимо учесть:
\begin{itemize}
  \item число требований - в них описываются функциональные возможности ПО
  \item число файлов исходного кода - в них на языке программирования реализованы требования
  \item число плагинов - они включают в себя файлыы исходного кода
  \item трассируемость требований на файлы исходного кода
  \item зависимости между файлами исходного кода
  \item распределение файлов исходного кода по плагинам
\end{itemize}

% На какие вопросы рассматриваемые ранее задачи не ответили
Благодаря этим сведениям можно построить метаматическую модель, отвечающую на вопрос, какие требования будут реализованы в поставке. Однако если имеющуюся информацию дополнить сведениями о стоимости сопровождения требований и механизмами формирования результирующей стоимости, тогда может быть решена задача оптимизации внутренней структуры программного решения по критерию минимальной стоимости ППО.
