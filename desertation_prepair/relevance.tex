% Актуальность темы исследования
\textbf{Актуальность темы исследования}

% Что сказать:
% - Инструментальные средства конфигурирования важны
% - Способ организации их в плагинных системах тоже важен
% - Решение задачи об оптимальном распределении функционала по плагинам тоже. ТАк вот о нем...

В настоящее время информационные технологии активно развиваются. Их развитие сопровождается неминуемым усложнением как самих информационных решений, так и технологий, применяемых для создания конечного продукта. Примером таких технологий являются инструментальные средства конфигурирования. С их помощью специалисты могут настраивать поведение информационных систем в допустимой конфигурации:
\begin{itemize}
    \item для операционных систем - настройка системных ресурсов, энергозависимой и энергонезависимосй памяти, расписание работы компонентов системы;
    \item для прикладных программ - их поведение, правила журналирования своей работы, поддержка различных протоколов взаимодействия;
    \item для баз данных - управление доступом, шардирование и реплекация;
    \item и т.д.
\end{itemize}

В рамках одной предметной области объем решаемых задач разными специалистами может существенно отличаться. В следствие этого отличается и состав полезного функционала, используемого в инструментальном средстве конфигурирования.

Удовлетворение спроса на инструментальные программные средства с разным объемом функционала может быть реализовано поставкой решения в различных комплектациях.

Это может быть достигнуто применением технологии плагинных систем. В них результирующий объем функционала зависит от состава установленных расширений системы - плагинов.

Применение плагинов для решения задачи поставки функционала в различных комплектациях не лишено недостатков. Одним из недостатков является необходимость поставки бесполезного для заказчика функционала в следствие технической необходимости разрешения функциональных зависимостей.

Анализ показал, что на сегодняшний день применяются следующие методы построения решения в плагинной системе:
\begin{enumerate}
    \item разместить весь функционал в одном плагине - очевидно не обеспечивает возможность поставки функционала в различных комплектациях
    \item разместить каждую единицу функционала в своем плагине - при достаточно большом числе плагинов становится невожным управление ими
    \item разместить функционал по плагинам по некоторому критерию - существуют разные подходы к выбору критерия, в рамках диссертационного исследования сформулирован и описан один из них.
\end{enumerate}

В рамках диссертационного исследования рассматривается задача поиска оптимальной декомпозиции функционала по плагинам. Критерием оптимальности распределения является минимизация стоимости постпродажного обслуживания. В рассматриваемой задаче стоимость постпродажного обслуживания поставки, которая формируется в соответствии с заявленной комплектацией, зависит от состава вошедших в поставку требований. Стоимость постпродажного обслуживания поставки рассчитывается как сумма стоимостей постпродажного обслуживания каждого из реализованных в поставке требований. При этом стоимость постпродажного обслуживания каждого отдельного требования зависит как от себестоимости его сопровождения так и от того, какие требования дополнительно были включены в поставку: включение их может как повышать, так и понижать стоимость.